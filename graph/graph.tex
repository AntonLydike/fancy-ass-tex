\documentclass[12pt]{article}

\usepackage{pgfplots}

% \pgfplotsset{compat=1.15}
\pgfplotsset{grid style={draw=none}}

\newcommand{\belldividers}[2]{
  \addplot [black] coordinates {
    (-#1,0)
    (-#1,#2)
  } node[below,pos=0] {-#1};
  \addplot [black] coordinates {
    (#1,0)
    (#1,#2)
  } node[below,pos=0] {#1};
}


\begin{document}

Some fancy-ass latex function:

\bigskip


\begin{figure}[!htbp]

  \pgfmathdeclarefunction{gauss}{2}{%
    \pgfmathparse{1/(#2*sqrt(2*pi))*exp(-((x-#1)^2)/(2*#2^2))}%
  }

  \centering
  \resizebox {\columnwidth} {!} {

    \begin{tikzpicture}
      \begin{axis}[
          no markers, domain=-3:3, samples=100,
          y axis line style={draw=none},
          xtick=\empty, ytick=\empty,
          axis lines*=left, xlabel=$z$, ylabel=$f(z)$,
          every axis y label/.style={at=(current axis.above origin),anchor=south},
          every axis x label/.style={at=(current axis.right of origin),anchor=west},
          height=6cm, width=1.1\textwidth,
          enlargelimits=false, clip=false, axis on top
        ]
        % greyed out area
        \addplot [fill=black!20, draw=none, domain=-1.96:1.96] {gauss(0,1)} \closedcycle;
        % graph
        \addplot [black] {gauss(0,1)};
        % lines
        % center line
        \addplot [black] coordinates {
          (0,0)
          (0,0.4)
        } node[pos=0.3] {95\%};
        % outer most lines
        % f(2.58) = 0.0143
        \belldividers{2.58}{0.0143}

        % f(1.96) = 0.05844
        \belldividers{1.96}{0.05844}

        % f(1) = 0.2419
        \belldividers{1}{0.2419}
        % add horizontal line
        \addplot [black] coordinates {
          (-1,0.2419)
          (0,0.2419)
        } node[above,pos=0.5] {$\sigma_z = 1$};
        \addplot [black] coordinates {
          (0,0.2419)
          (1,0.2419)
        } node[above,pos=0.5] {$\sigma_z = 1$};

        %\draw [yshift=-0.4cm, latex-latex](axis cs:-1,0) -- node [fill=white] {$\sigma_z = 1$} (axis cs:0,0);
      \end{axis}
    \end{tikzpicture}
  }
  \end{figure}



\end{document}


\end{document}
